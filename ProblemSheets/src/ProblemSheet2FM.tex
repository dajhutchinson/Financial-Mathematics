\documentclass[11pt,a4paper]{article}

\usepackage[margin=1in, paperwidth=8.3in, paperheight=11.7in]{geometry}
\usepackage{amsmath,amsfonts,fancyhdr,bbm}
\usepackage[section,nohyphen]{DomH}
\headertitle{Financial Mathematics - Problem Sheet 2}

\newcommand\aug{\fboxsep=-\fboxrule\!\!\!\fbox{\strut}\!\!\!}

\begin{document}

% \questionsfalse
% \answersfalse

\title{Financial Mathematics - Problem Sheet 2}
\author{Dom Hutchinson}
\date{\today}
\maketitle

\begin{question}{2.}
  Consider a single-period model with one risky asset that has a current price of $S_0=15$ and a future price of either $S_1(\omega_1)=20,\ S_1(\omega_2)=40/3$ or $S_1(\omega_3)=10$. There is also a bank account with a constant interest rate $r=1/9$.
\end{question}

\begin{question}{2. a)}
  Specify the value and gains processes $V,G$ for this model as well as their discounted versions $V^*,G^*$.
\end{question}

\begin{answer}{2. a)}
  When using a trading strategy $H:=(H_0,H_1)$, this model has value process $V$ and gains process $G$, defined as
  \[\begin{array}{rrrl}
    &V&:=&\{V_t:t=0,1\}\\
    \text{where }&V_t&:=&H_0(1+rt)+H_1\cdot S_1(t)\\
    &&=&H_0\left(1+\frac{t}9\right)+H_1\cdot S_1(t)\\
    \\
    &G&=&H_0r+H_1\Delta S_1(t)\\
    &&=&\frac19H_0+(S_1(t)-15)H_1
  \end{array}\]
  Note that in this model $B_t=1+rt$ with $r=1/9$. Thus the discounted value process $V^*$ and discounted gains process $G^*$ for this model are defined as
  \[\begin{array}{rrcl}
    &V^*&:=&\{V_t^*:t=0,1\}\\
    \text{where }&V_t^*&:=&\frac{V_t}{B_t}\\
    &&=&\frac{V_t}{1+rt}\\
    \\
    &G^*&:=&\frac{G}{B_1}\\
    &&=&\frac{G}{1+r}
  \end{array}\]
\end{answer}

\begin{question}{2. b)}
  By describing the discounted price process $S^*$, and using the condition that $\expect_\Q[S^*(1)]=S^*(0)$, determine all risk neutral probability measures $\Q$.
\end{question}

\begin{answer}{2. b)}
  For this model, the discounted price process $S^*$ is defined as
  \[\begin{array}{rrrl}
    &S^*&:=&\{S_1^*(t):t=0,1\}\\
    \text{where}&S_1^*(t)&:=&\frac{S_1(t)}{B_t}\\
    &&=&\frac{S_1(t)}{1+rt}
  \end{array}\]
  Define a probability measure $\Q$ and let $q_i:=\Q(\{\omega_i\})$ for $i\in\{1,2,3\}$. Consider both sides of the expression $\expect_\Q[S_1^*(1)]=S_1^*(0)$.
  \[\begin{array}{rcl}
    S_1^*(0)&=&\frac{S_1(0)}{B_0}\\
    &=&15\\
    \expect_\Q[S_1^*(1)]&=&\sum_{i=1}^3q_iS^*(1)(\omega_i)\\
    &=&\sum_{i=1}^3q_i\frac{S(1)(\omega_i)}{B_1}\\
    &=&\frac1{1+r}(20q_1+(40/3)q_2+10q_3)
  \end{array}\]
  For $\Q$ to be a risk-neutral probability measure, the following must all hold
  \begin{enumerate}
    \item $\Q(\{\omega_i\})>0\ \forall\ i\in\{1,2,3\}$.
    \item $\sum_{i=1}^3\Q(\{\omega_i\})=1$.
    \item $\expect_\Q[S_1^*(1)]=S_1^*(0)\implies\frac1{1+r}\left(20q_1+(40/3)q_2+10q_3\right)=15$.
  \end{enumerate}
  We can state, and find solutions to, the last two conditions as the following matrix
  \[\begin{array}{rrcl}
    &&&\begin{pmatrix}
      1&1&1&\aug&1\\
      \frac{20}{1+r}&\frac{40}{3(1+r)}&\frac{10}{1+r}&\aug&15
    \end{pmatrix}\\
    &&=&\begin{pmatrix}
      1&1&1&\aug&1\\
      18&12&9&\aug&15
    \end{pmatrix}\\
    &&=&\begin{pmatrix}
      1&1&1&\aug&1\\
      6&4&3&\aug&5
    \end{pmatrix}\\
    &&=&\begin{pmatrix}
      1&1&1&\aug&1\\
      3&1&0&\aug&2
    \end{pmatrix}\\
    \implies&q_3&=&1-q_1-q_2\\
    \text{and}&q_2&=&2-3q_1\\
    \implies&q_3&=&1-q_1-(2-3q_1)\\
    &&=&2q_1-1\\
    \implies&\mathbf{q}&=&\begin{pmatrix}
      q_1&2-3q_1&2q_1-1
    \end{pmatrix}
  \end{array}\]
  In order for the first condition to be fulfilled we require
  \[\begin{array}{rrcl}
    &q_1&\in&[0,1]\\
    &q_2=2-3q_1&\in&[0,1]\\
    \implies&q_1&\in&[1/3,2/3]\\
    &q_3=2q_1-1&\in&[0,1]\\
    \implies&q_1&\in&[0,1]
  \end{array}\]
  Thus, for the first condition to be true we must restrict $q_1$ to values in $[1/3,2/3]$.
  \par To summarise, $\Q$ is a risk-neutral probability measure for this model if
  \[ \Q(\{\omega_1\})=q_1\quad\Q(\{\omega_2\})=2-3q_1\quad\Q(\{\omega_3\})=2q_1-1\quad\text{for any }q_1\in[1/3,2/3] \]
\end{answer}

\begin{question}{2. c)}
  Is the market complete? Describe the set of attainable contingent claims $X=(X_1,X_2,X_3)$ for this model.
\end{question}

\begin{answer}{2. c)}
  Consider the following matrix $A$ which summarises this market
  \[ A=\begin{pmatrix}
    B_1(\omega_1)&S_1(1)(\omega_1)\\
    B_1(\omega_2)&S_1(1)(\omega_2)\\
    B_1(\omega_3)&S_1(1)(\omega_3)
  \end{pmatrix}=\begin{pmatrix}
    10/9&20\\
    10/9&40/3\\
    10/9&10
  \end{pmatrix}
  \]
  Matrix $A$ only has two linearly-independent columns (ie $\text{rank}(A)=2$), this means $AH$ can only span $R^2$ for any $H$. Thus, this market is \underline{not} complete.
  \par Under this model, a contingent claim $X$ can have any value for two of its three dimensions, but the value of the third dimension has to fulfil a specific relationship with the other two in order for $X$ to be attainable. This dependency is defined in the requirement that $\exists\ H\in\reals^2$ st $AH=X$ for $X$ to be attainable
\end{answer}

\begin{question}{3.}
  Consider a single-period model with two risky securities $S_1$ and $S_2$. Assume that there is a bank account with interest rate $r=1/9$ and that the price process is given in the following table
  \begin{center}
    \begin{tabular}{c|c|ccc}
      $n$&$S_n(0)$&$S_n(1)(\omega_1)$&$S_n(1)(\omega_2)$&$S_n(1)(\omega_3)$\\\hline
      1&15&20&20&40/3\\
      2&30&40&80/3&80/3
    \end{tabular}
  \end{center}
\end{question}

\begin{question}{3. a)}
  Specify the value and gains processes $V,G$ for this model as well as their discounted versions $V^*,G^*$.
\end{question}

\begin{answer}{3. a)}
  When using a trading strategy $H:=(H_0,H_1,H_2)$, this model has value process $V$ and gains process $G$, defined as
  \[\begin{array}{rrrl}
    &V&:=&\{V_t:t=0,1\}\\
    \text{where }&V_t&:=&H_0(1+rt)+H_1\cdot S_1(t)\\
    &&=&H_0\left(1+\frac{t}9\right)+H_1\cdot S_1(t)+H_2\cdot S_2(t)\\
    \\
    &G&=&H_0r+H_1\Delta S_1(t)+H_2\Delta S_2(t)\\
    &&=&\frac19H_0+(S_1(t)-15)H_1+(S_2(t)-30)H_2
  \end{array}\]
  Note that in this model $B_t=1+rt$ with $r=1/9$. Thus the discounted value process $V^*$ and discounted gains process $G^*$ for this model are defined as
  \[\begin{array}{rrcl}
    &V^*&:=&\{V_t^*:t=0,1\}\\
    \text{where }&V_t^*&:=&\frac{V_t}{B_t}\\
    &&=&\frac{V_t}{1+rt}\\
    \\
    &G^*&:=&\frac{G}{B_1}\\
    &&=&\frac{G}{1+r}
  \end{array}\]
\end{answer}

\begin{question}{3. b)}
  By considering the two discounted stock processes $S_1^*,S_2^*$, show that there is no risk neutral probability measure.
\end{question}

\begin{answer}{3. b)}
  For this model, the discounted price processes $S_1^*$ and $S_2^*$ are defined as
  \[\begin{array}{rcl}
    S_1^*(t)&:=&\frac{S_1(t)}{B_t}=\frac{S_1(t)}{1+rt}\\
    S_2^*(t)&:=&\frac{S_2(t)}{B_t}=\frac{S_2(t)}{1+rt}
  \end{array}\]
  Let $\Q$ be a probability measure and defined $q_i:=\Q(\{\omega_i\})$ for $i\in\{1,2,3\}$. For $\Q$ to be a risk-neutral probability measure, the following must all hold
  \begin{enumerate}
    \item $q_i>0\ \forall\ i\in\{1,2,3\}$.
    \item $q_1+q_2+q_3=1$.
    \item $\expect_\Q[S_1^*(1)]=S_1^*(0)\implies\frac1{1+r}\left(20q_1+20q_2+(40/3)q_3\right)=15$.
    \item $\expect_\Q[S_2^*(1)]=S_2^*(0)\implies\frac1{1+r}\left(40q_1+(80/3)q_2+(80/3)q_3\right)=30$.
  \end{enumerate}
  We can express the last three equations in the following matrix % TODO
  \[\begin{array}{rl}
    &\begin{pmatrix}
        1&1&1&\aug&1\\
        \frac{20}{1+r}&\frac{20}{1+r}&\frac{40}{3(1+r)}&\aug&15\\
        \frac{40}{1+r}&\frac{80}{3(1+r)}&\frac{80}{3(1+r)}&\aug&30\\
      \end{pmatrix}\\
    =&\begin{pmatrix}
        1&1&1&\aug&1\\
        18&18&12&\aug&15\\
        36&24&24&\aug&30\\
    \end{pmatrix}\\
    =&\begin{pmatrix}
        1&1&1&\aug&1\\
        0&0&-6&\aug&-3\\
        12&0&0&\aug&6
    \end{pmatrix}\\
    =&\begin{pmatrix}
        1&1&1&\aug&1\\
        0&0&1&\aug&1/2\\
        1&0&0&\aug&1/2
    \end{pmatrix}\\
    =&\begin{pmatrix}
        0&1&0&\aug&0\\
        0&0&1&\aug&1/2\\
        1&0&0&\aug&1/2
    \end{pmatrix}\\
  \end{array}\]
  This shows there is a unique solution to this system of equations $(q_1,q_2,q_3)=(0,1/2,1/2)$, but this violates the first condition for $\Q$ to be a risk-neutral probability measure. Thus, there are no risk-neutral probability measures for this model.
\end{answer}

\begin{question}{3. c)}
  Determine expressions for $\mathbb{W}$ and $\mathbb{W}\cap\mathbb{A}$ where
  \[\begin{array}{rcl}
    \mathbb{W}&:=&\left\{X\in\reals^K:X=G^*\text{ for some strat. }H\right\}\\
    \mathbb{A}&:=&\left\{X\in\reals^K:X\geq0,X\neq0\right\}
  \end{array}\]
  Determine all arbitrage opportunities by deriving for any $X\in\mathbb{W}\cap\mathbb{A}$ a trading strategy $H$ which gives rise to the time $t=1$ portfolio value $(X_1,X_2,X_3)$.
\end{question}

\begin{answer}{3. c)}
  Consider generic $X,H\in\reals^3$ and define matrix $A\in\reals^{3\times3}$ as
  \[ A=\begin{pmatrix}
    B_1(\omega_1)&S_1(1)(\omega_1)&S_2(1)(\omega_1)\\
    B_1(\omega_2)&S_1(1)(\omega_2)&S_2(1)(\omega_2)\\
    B_1(\omega_3)&S_1(1)(\omega_3)&S_2(1)(\omega_3)
  \end{pmatrix}=\begin{pmatrix}
    1+r&20&40\\
    1+r&20&80/3\\
    1+r&40/3&80/3
  \end{pmatrix}=\begin{pmatrix}
    10/9&20&40\\
    10/9&20&80/3\\
    10/9&40/3&80/3
  \end{pmatrix} \]
  The strategy $H$ attains contingent claim $X$ if $AH=X$. Thus
  \[\begin{array}{rcl}
    H&=&A^{-1}X\\
    \begin{pmatrix}
      H_1\\H_2\\H_3
    \end{pmatrix}&=&
    \begin{pmatrix}
      -9/5&0&27/10\\
      0&3/20&-3/20\\
      3/40&-3/40&0
    \end{pmatrix}\begin{pmatrix}
      X_1\\X_2\\X_3
    \end{pmatrix}\\
    &=&\begin{pmatrix}
        (9/10)(3X_3-2X_1)\\
        (3/20)(X_2-X_3)\\
        (3/40)(X_1-X_2)
      \end{pmatrix}
  \end{array}\]
  This gives a formula for what trading strategy $H$ to give rise to a portfolio of value $(X_1,X_2,X_3)$ at time $t=1$.
\end{answer}

\end{document}
