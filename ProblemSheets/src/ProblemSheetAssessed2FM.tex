\documentclass[11pt,a4paper]{article}

\usepackage[margin=1in, paperwidth=8.3in, paperheight=11.7in]{geometry}
\usepackage{amsmath,amsfonts,fancyhdr,bbm}
\usepackage[section,nohyphen]{DomH}
\headertitle{Financial Mathematics - Assessed Problem Sheet 2}

\begin{document}

\questionsfalse
% \answersfalse

\title{Financial Mathematics - Assessed Problem Sheet 2}
\author{Dom Hutchinson}
\date{\today}
\maketitle

\begin{question}{1. (a)}
  Consider a model with two periods and one risky security where the interest rate is $r=0$ and the price process is given by the following table
  \begin{center}
    $S(t)(\omega)=$
    \begin{tabular}{c|ccc}
      $\omega\setminus t$&0&1&2\\\hline
      $\omega_1$&10&16&18\\
      $\omega_2$&10&16&12\\
      $\omega_3$&10&8&12\\
      $\omega_4$&10&8&6
    \end{tabular}
  \end{center}
  Consider the following two options
  \begin{enumerate}
    \item An \textit{Asain Option} with strike price 14 and expiry time 2, so that its payoff is
    \[ X=\left\{\frac13\left(S(0)+S(1)+S(2)\right)-14\right\}_+ \]
    \item An \textit{Look-Back Option} with strike price 14 and expiry time 2, so that its payoff is
    \[ X=\left\{\max_{t=0,1,2}S(t)-14\right\}_+ \]
  \end{enumerate}
\end{question}

\begin{question}{1. (a) i.}
  Calculate the risk-neutral probability measure $\Q$
\end{question}

\begin{answer}{1. (a) i.}
  Consider time $t=0$ and define $p_1:=\Q(S_1=16)$, then
  \[\begin{array}{rrcl}
    &10&=&16p_1+8(1-p_1)\\
    \implies&p_1&=&1/4
  \end{array}\]
  Thus $\Q(S_1=16)=1/4$ and $\Q(S_1=8)=1-1/4=3/4$.
  Now, consider time $t=1$ and that event $\omega$ has occurred with $\omega\in\{\omega_1,\omega_2\}$. Define $p_2:=\Q(S_2=18|S_1=16)$, then
  \[\begin{array}{rrcl}
    &16&=&18p_2+12(1-p_2)\\
    \implies&p_2&=&2/3
  \end{array}\]
  Thus $\Q(S_2=18|S_1=16)=2/3$ and $\Q(S_2=12|S_1=16)=1/3$.
  Now, consider time $t=1$ and that event $\omega$ has occurred with $\omega\in\{\omega_3,\omega_4\}$. Define $p_3:=\Q(S_2=12|S_1=8)$, then
  \[\begin{array}{rrcl}
    &8&=&12p_3+6(1-p_3)\\
    \implies&p_3&=&1/3
  \end{array}\]
  Thus $\Q(S_2=12|S_1=8)=1/3$ and $\Q(S_2=6|S_1=8)=2/3$.
  \par We can use these conditional probabilities to work out the probability of each event $\{\omega_1,\omega_2,\omega_3,\omega_4\}$ under $\Q$.
  \[\begin{array}{rcl}
    \Q(\{\omega_1\})&=&\Q(S_2=18|S_1=16)\Q(S_1=16)\\
    &=&(2/3)\times(1/4)\\
    &=&1/6\\
    \Q(\{\omega_2\})&=&\Q(S_2=12|S_1=16)\Q(S_1=16)\\
    &=&(1/3)\times(1/4)\\
    &=&1/12\\
    \Q(\{\omega_3\})&=&\Q(S_2=12|S_1=8)\Q(S_1=8)\\
    &=&(1/3)\times(3/4)\\
    &=&1/4\\
    \Q(\{\omega_4\})&=&\Q(S_2=6|S_1=8)\Q(S_1=8)\\
    &=&(2/3)\times(3/4)\\
    &=&1/2
  \end{array}\]
  To confirm that this produces a probability measure, note that
  \[ (1/6)+(1/12)+(1/4)+(1/2)=1 \]
\end{answer}

\begin{question}{1. (a) ii.}
  Determine $X(\omega_i),\ Y(\omega_i)$ for $i=1,2,3,4$.
\end{question}

\begin{answer}{1. (a) ii.}
  \[\begin{array}{rcl}
    X(\omega_1)&=&\left\{\frac13(10+16+18)-14\right\}_+\\
    &=&\left\{\frac{44}3-14\right\}_+\\
    &=&2/3\\
    X(\omega_2)&=&\left\{\frac13(10+16+12)-14\right\}_+\\
    &=&\left\{\frac{38}3-14\right\}_+\\
    &=&0\\
    X(\omega_3)&=&\left\{\frac13(10+8+12)-14\right\}_+\\
    &=&\left\{\frac{30}3-14\right\}_+\\
    &=&0\\
    X(\omega_4)&=&\left\{\frac13(10+8+6)-14\right\}_+\\
    &=&\left\{\frac{24}3-14\right\}_+\\
    &=&0\\\\
    Y(\omega_1)&=&\left\{\max(10,16,18)-14\right\}_+\\
    &=&\left\{18-14\right\}_+\\
    &=&4\\
    Y(\omega_2)&=&\left\{\max(10,16,12)-14\right\}_+\\
    &=&\left\{16-14\right\}_+\\
    &=&2\\
    Y(\omega_3)&=&\left\{\max(10,8,12)-14\right\}_+\\
    &=&\left\{12-14\right\}_+\\
    &=&0\\
    Y(\omega_4)&=&\left\{\max(10,8,6)-14\right\}_+\\
    &=&\left\{10-14\right\}_+\\
    &=&0
  \end{array}\]
\end{answer}

\begin{question}{1. (a) iii.}
  Determine the risk-neutral prices of $X$ and $Y$ at time $t=0$.
\end{question}

\begin{answer}{1. (a) iii.}
  The \textit{Risk-Neutral Valuation Principle} states that, for all attainable contingent claims $X$, the following holds
  \[ V_t^*=\expect_Q[X/B_t|\mathcal{F}_t]\text{ for }t=0,\dots,T\text{ and all }\Q \]
  Thus, at time $t=0$
  \[ V_0=\expect_\Q[X|\mathcal{F}_0]=\expect_\Q[X] \]
  We can use this to derive the risk-neutral prices for the two options at time $t=0$
  \[\begin{array}{rcl}
    V_0^X&=&\expect_\Q[X]\\
    &=&(2/3)\cdot(1/6)+0\cdot(1/12)+0\cdot(1/4)+0\cdot(1/2)\\
    &=&1/9\\\\
    V_0^Y&=&\expect_\Q[Y]\\
    &=&4\cdot(1/6)+2\cdot(1/12)+0\cdot(1/4)+0\cdot(1/2)\\
    &=&5/6
  \end{array}\]
  Thus the risk-neutral price for the Asian option at time $t=0$ is $1/9$ and for the look-back option it is $5/6$
\end{answer}

\begin{question}{1. (a) iv.}
  Find self-financing trading strategies for each of the claims $X$ and $Y$.
\end{question}

\begin{answer}{1. (a) iv.}
  Note that since $r=0,\ B_t=1$ for $i=0,1,2$.
  \par Consider the Asian option and let $H^X(t):=\{H_0^Y(t),H_1^Y(t)\}$ denote a portfolio which only has access to a bank account \& Asian options for each time point $t=0,1,2$.
  \par consider the value of the Asian option at time $t=1$ and $t=2$
  \[\begin{array}{rcl}
    V_1^X(\omega)&=&\begin{cases}
      \expect[X|S_1=16]&\text{if }\omega\in\{\omega_1,\omega_2\}\\
      \expect[X|S_1=8]&\text{if }\omega\in\{\omega_3,\omega_4\}
    \end{cases}\\
    &=&\begin{cases}
      (2/3)\cdot p_2+0\cdot(1-p_2)&\text{if }\omega\in\{\omega_1,\omega_2\}\\
      0&\text{if }\omega\in\{\omega_3,\omega_4\}
    \end{cases}\\
    &=&\begin{cases}
      (2/3)\cdot(2/3)&\text{if }\omega\in\{\omega_1,\omega_2\}\\
      0&\text{if }\omega\in\{\omega_3,\omega_4\}
    \end{cases}\\
    &=&\begin{cases}
      4/9&\text{if }\omega\in\{\omega_1,\omega_2\}\\
      0&\text{if }\omega\in\{\omega_3,\omega_4\}
    \end{cases}\\
    V_2^X(\omega)&=&X(\omega)\ \forall\ \omega
  \end{array}\]
  To find a self-financing portfolio we start at time $t=2$ and find that if $\omega\in\{\omega_1,\omega_2\}$
  \[\begin{array}{rrcl}
    &V_2^X(\omega_1)=2/3&=&H_0^X(2)(\omega_1,\omega_2)+18\cdot H_1^X(2)(\omega_1,\omega_2)\\
    &V_2^X(\omega_2)=0&=&H_0^X(2)(\omega_1,\omega_2)+12\cdot H_1^X(2)(\omega_1,\omega_2)\\\\
    \implies&2/3&=&6H_1^X(2)(\omega_1,\omega_2)\\
    \implies&H_1^X(2)(\omega_1,\omega_2)&=&1/9\\\\
    \implies&2&=&H_0^X(2)(\omega_1,\omega_2)+12\cdot(1/9)\\
    \implies&H_0^X(2)(\omega_1,\omega_2)&=&-4/3
  \end{array}\]
  And if $\omega\in\{\omega_3,\omega_4\}$
  \[\begin{array}{rrcl}
    &V_2^X(\omega_3)=0&=&H_0^X(2)(\omega_3,\omega_4)+12\cdot H_1^X(2)(\omega_3,\omega_4)\\
    &V_2^X(\omega_4)=0&=&H_0^X(2)(\omega_3,\omega_4)+6\cdot H_1^X(2)(\omega_3,\omega_4)\\\\
    \implies&H_0^X(2)(\omega_3,\omega_4)&=&0\\
    \implies&H_1^X(2)(\omega_3,\omega_4)&=&0
  \end{array}\]
  Let $H_0^X(1):=H_0^X(1)(\omega_1,\omega_2,\omega_3,\omega_4)$ and $H_1^X(1):=H_1(1)(\omega_1,\omega_2,\omega_3,\omega_4)$, then we have
  \[\begin{array}{rrcll}
    &V_1^X(\omega)=4/9&=&H_0^X(1)+16\cdot H_1^X(1)&\text{ if }\omega\in\{\omega_1,\omega_2\}\\
    &V_1^X(\omega)=0&=&H_0^X(1)+8\cdot H_1^X(1)&\text{ if }\omega\in\{\omega_3,\omega_4\}\\\\
    \implies&4/9&=&8H_1^X(1)\\
    \implies&H_1^X(1)&=&1/18\\\\
    \implies&0&=&H_0^X(1)+8\cdot(1/18)\\
    \implies&H_0^X(1)&=&-4/9
  \end{array}\]
  Thus, a self-financing trading strategy with access to a bank account and the look-back option is as follows
  \[\begin{array}{rcll}
    H^X(1)(\omega)&=&(-4/9,1/18)&\forall\ \omega\\
    H^X(2)(\omega)&=&(-4/3,1/9)&\text{if }\omega\in\{\omega_1,\omega_2\}\\
    H^X(2)(\omega)&=&(0,0)&\text{if }\omega\in\{\omega_3,\omega_4\}
  \end{array}\]
  To confirm this trading strategy is self-financing I show that $H_0(t+1)B_t+H_1(t+1)S_1(t)=V_t\ \forall\ t$.
  \[\begin{array}{r|rclcll}
    H^Y(1)(\omega)&B_0\cdot H_0^X(1)(\omega)+S_1(0)\cdot H_1^Y(1)(\omega)&=&-(4/9)+10\cdot(1/18)\\
    &&=&1/9=V_0^X\\
    H^Y(2)(\omega)&B_1\cdot H_0^X(2)(\omega)+S_1(1)\cdot H_1^Y(2)(\omega)&=&-(4/3)+16\cdot(1/9)&\text{if }\omega\in\{\omega_1,\omega_2\}\\
    &&=&4/9=V_1^Y(\omega)\\
    H^Y(2)(\omega)&B_1\cdot H_0^X(2)(\omega)+S_1(1)\cdot H_1^Y(2)(\omega)&=&0+8\cdot0&\text{if }\omega\in\{\omega_3,\omega_4\}\\
    &&=&0=V_1^Y(\omega)
  \end{array}\]
  This requirement holds in all cases.
  \vspace{5mm}
  \par Now, consider the look-back option and let $H^Y(t):=\{H_0^Y(t),H_1^Y(t)\}$ denote a portfolio which only has access to a bank account \& look-back options for each time point $t=0,1,2$.
  \par Consider the value of the look-back option at time $t=1$ and $t=2$
  \[\begin{array}{rcl}
    V_1^Y(\omega)&=&\begin{cases}
      \expect[Y|S_1=16]&\text{if }\omega\in\{\omega_1,\omega_2\}\\
      \expect[Y|S_1=8]&\text{if }\omega\in\{\omega_3,\omega_4\}
    \end{cases}\\
    &=&\begin{cases}
      4p_2+2(1-p_2)&\text{if }\omega\in\{\omega_1,\omega_2\}\\
      0&\text{if }\omega\in\{\omega_3,\omega_4\}
    \end{cases}\\
    &=&\begin{cases}
      4\cdot(2/3)+2(1/3)&\text{if }\omega\in\{\omega_1,\omega_2\}\\
      0&\text{if }\omega\in\{\omega_3,\omega_4\}
    \end{cases}\\
    &=&\begin{cases}
      10/3&\text{if }\omega\in\{\omega_1,\omega_2\}\\
      0&\text{if }\omega\in\{\omega_3,\omega_4\}
    \end{cases}\\
    V_2^Y(\omega)&=&Y(\omega)\ \forall\ \omega
  \end{array}\]
  To find a self-financing portfolio we start at time $t=2$ and find that if $\omega\in\{\omega_1,\omega_2\}$
  \[\begin{array}{rrcl}
    &V_2^Y(\omega_1)=4&=&H_0^Y(2)(\omega_1,\omega_2)+18\cdot H_1^Y(2)(\omega_1,\omega_2)\\
    &V_2^Y(\omega_2)=2&=&H_0^Y(2)(\omega_1,\omega_2)+12\cdot H_1^Y(2)(\omega_1,\omega_2)\\\\
    \implies&2&=&6H_1^Y(2)(\omega_1,\omega_2)\\
    \implies&H_1^Y(2)(\omega_1,\omega_2)&=&1/3\\\\
    \implies&2&=&H_0^Y(2)(\omega_1,\omega_2)+12\cdot(1/3)\\
    \implies&H_0^Y(2)(\omega_1,\omega_2)&=&-2/3
  \end{array}\]
  And if $\omega\in\{\omega_3,\omega_4\}$
  \[\begin{array}{rrcl}
    &V_2^Y(\omega_3)=0&=&H_0^Y(2)(\omega_3,\omega_4)+12\cdot H_1^Y(2)(\omega_3,\omega_4)\\
    &V_2^Y(\omega_4)=0&=&H_0^Y(2)(\omega_3,\omega_4)+6\cdot H_1^Y(2)(\omega_3,\omega_4)\\\\
    \implies&H_0^Y(2)(\omega_3,\omega_4)&=&0\\
    \implies&H_1^Y(2)(\omega_3,\omega_4)&=&0
  \end{array}\]
  Let $H_0^Y(1):=H_0(1)(\omega_1,\omega_2,\omega_3,\omega_4)$ and $H_1^Y(1):=H_1(1)(\omega_1,\omega_2,\omega_3,\omega_4)$, then we have
  \[\begin{array}{rrcll}
    &V_1^Y(\omega)=10/3&=&H_0^Y(1)+16\cdot H_1^Y(1)&\text{ if }\omega\in\{\omega_1,\omega_2\}\\
    &V_1^Y(\omega)=0&=&H_0^Y(1)+8\cdot H_1^Y(1)&\text{ if }\omega\in\{\omega_3,\omega_4\}\\\\
    \implies&10/3&=&8H_1^Y(1)\\
    \implies&H_1^Y(1)&=&5/12\\\\
    \implies&0&=&H_0^Y(1)+8\cdot(5/12)\\
    \implies&H_0^Y(1)&=&-40/12
  \end{array}\]
  Thus, a self-financing trading strategy with access to a bank account and the look-back option is as follows
  \[\begin{array}{rcll}
    H^Y(1)(\omega)&=&(-40/12,5/12)&\forall\ \omega\\
    H^Y(2)(\omega)&=&(-2/3,1/3)&\text{if }\omega\in\{\omega_1,\omega_2\}\\
    H^Y(2)(\omega)&=&(0,0)&\text{if }\omega\in\{\omega_3,\omega_4\}
  \end{array}\]
  To confirm this trading strategy is self-financing I show that $H_0(t+1)B_t+H_1(t+1)S_1(t)=V_t\ \forall\ t$.
  \[\begin{array}{r|rclcll}
    H^Y(1)(\omega)&B_0\cdot H_0^Y(1)(\omega)+S_1(0)\cdot H_1^Y(1)(\omega)&=&-(40/12)+10\cdot(5/12)\\
    &&=&5/6=V_0^Y\\
    H^Y(2)(\omega)&B_1\cdot H_0^Y(2)(\omega)+S_1(1)\cdot H_1^Y(2)(\omega)&=&-(2/3)+16\cdot(1/3)&\text{ if }\omega\in\{\omega_1,\omega_2\}\\
    &&=&10/3=V_1^Y(\omega)\\
    H^Y(2)(\omega)&B_1\cdot H_0^Y(2)(\omega)+S_1(1)\cdot H_1^Y(2)(\omega)&=&0+8\cdot0&\text{ if }\omega\in\{\omega_3,\omega_4\}\\
    &&=&0=V_1^Y(\omega)
  \end{array}\]
  This requirement holds in all cases.
\end{answer}

\begin{question}{1. (b)}
  Consider a multi-period model with $N=1$ risky securities and a finite probability space $\Omega=\{\omega_1,\dots,\omega_K\}$. For each state $\omega\in\Omega$, each time-point $t=0,\dots,T$ define $S(t)(\omega)$ to be the price of the security at time point $t$ in state $\omega$. We assume that there is a bank account with interest rate $r=0$.
  \par Let us assume that the price process satisfies
  \[\begin{array}{rcl}
    S(0)&=&1\\
    S(t)&=&\begin{cases}
      S(t-1)\exp\{X(t-1)\}&\text{if }S(t-1)>\epsilon\\
      S(t-1)\exp\{X(t-1)+2\}&\text{if }S(t-1)\leq\epsilon
  \end{cases}\text{ for }t=1,\dots,T
  \end{array}\]
  where $\epsilon\in(0,1)$ is a constant and $X:=\{X(t)\}_{t\geq0}$ is a sequence of IID random variables st each $X(t)$ takes value 1 or -1 with probability 1/2 (each).
  \par Construct an arbitrage opportunity for this model and verify all the conditions which define such an opportunity.
  \par \textit{Hint - For a complete answer, you need to state appropriate assumptions on $\epsilon$ and $T$.}
\end{question}

\begin{answer}{1. (b)}
  Assume that $\epsilon$ is known, $T$ is large and note that $B_t=1\ \forall\ t\in[1,T]$ since the interest rate is $r=0$.
  \par Consider the following trading strategy
  \begin{quote}
    If, at time-point $t\in(1,T)$ it is the case that $S(t)\leq\epsilon$ then:
    \begin{itemize}
      \item In time-point $t$ borrow £$S(t)$ from the bank-account and buy a unit of the security for £$S(t)$.
      \item In time-point $t+1$ sell your security for £$S(t+1)$ and place all of this revenue into the bank account (reimbursing the bank at the same time).
    \end{itemize}
    Otherwise, do nothing.
  \end{quote}
  This strategy can be expressed mathematically as
  \[\begin{array}{rcl}
    H(0)&=&(0,0)\\
    H(t+1)&=&\begin{cases}
      \big(H_0(t)-S(t),1\big)&\text{if }S(t)\leq\epsilon\\
      \big(H_0(t)+S(t),0\big)&\text{if }S(t-1)\leq\epsilon\\
      \big(H_0(t),0\big)&\text{otherwise}
  \end{cases}
  \end{array}\]
  For this specific strategy we need to further assume that $\epsilon>e^{1-T}$ as otherwise the main action could never be triggered.

  \par I will how justify that this trading strategy is an arbitrage opportunity.

  \par Consider the value of this portfolio at time-point $t=0$
  \[ V_0=H_0(0)\cdot B_0+H_1(0)\cdot S(1)=0\cdot1+0\cdot1=0 \]
  Thus the requirement that $V_0=0$ is fulfilled.

  \par Suppose that $S(t)\leq\epsilon$ but $S(t-1)>\epsilon$. This means that the price process must have decreased in value between these time steps and thus $S(t)=S(t-1)e^{-1}\implies S(t)e>\epsilon$.
  \par Now consider $S(t+1)$, we know that $S(t+1)=S(t)\exp\{X(t)+2\}$ by the definition of the price process and since $S(t)\leq\epsilon$. Since $X(t)\in\{-1,1\}$ then $X(t)+2\geq1$, meaning $S(t+1)\geq S(t)e>\epsilon$. This shows that the price process only ever goes below threshold $\epsilon$ for a single time-period at a time.

  \par Now, suppose the action is triggered at time-point $t$ and consider the price process over this and the next time-point

  \[\begin{array}{rrcl}
    &S(t)&\leq&\epsilon\\
    \implies&S(t+1)&=&S(t)\exp\{X(t)+2\}\\
    \implies&S(t+1)&>&S(t)\text{ since }X(t)+2\geq1
  \end{array}\]

  This shows that when we sell the security in time-point $t+1$ it is always worth more than the price we bought it for, meaning we can fully reimburse the bank-account and have a surplus left over in our bank-account. Moreover, this shows that we are guaranteed to make a profit whenever the main action of this strategy is triggered. And, since this strategy takes no other actions, it cannot lose money between time-periods.
  \par Since we assumed that $T$ is large and since $\{X_t\}$ is a symmetric random walk it is expected that the action will occur at least once. This means the criteria that $V_T=H_0(T)\geq0$ and $\expect[V_T]>0$ are fulfilled.

  \par To show that this trading strategy $H$ is self-financing I need to show that $V_t(\omega)=H_0(t+1)+H_1(t+1)S(t)$ for all $t,\omega$. There are three cases to consider
  \begin{enumerate}
    \item Case $S(t)\leq\epsilon\implies S(t-1)>\epsilon$ (Shown above).
    \par No action was taken in the last time-period. Thus $V_t=H_0(t)$.
    \[\begin{array}{rcl}
      &&H_0(t+1)+H_1(t+1)S(t)\\
      &=&\big(H_0(t)-S(t)\big)+S(t)\\
      &=&H_0(t)\\
      &=&V_t
    \end{array}\]

    \item Case $S(t-1)\leq\epsilon\implies S(t)>\epsilon$ (Shown above).
    \par This means an option was bought in the last-time period for $S(t-1)$ and that option is now worth $S(t)$, thus $V_t=H_0(t)-S(t-1)+S(t)$.
    \[\begin{array}{rcl}
      &&H_0(t+1)+H_1(t+1)S(t)\\
      &=&\big(H_0(t)+S(t)\big)+0\\
      &=&H_0(t)+S(t)\\
      &=&(H_0(t-1)-S(t-1))+S(t)\text{ as }S(t-1)\leq\epsilon\\
      &=&V_t
    \end{array}\]

    \item Otherwise (ie $S(t),S(t-1)>\epsilon$)
    \par No actions are taken so $V_t=H_0(t)$.
    \[\begin{array}{rcl}
      &&H_0(t+1)+H_1(t+1)S(t)\\
      &=&H_0(t)+0\\
      &=&V_t
    \end{array}\]
  \end{enumerate}
  This shows that $H$ is self-financing, and thus fulfils all criteria to be an arbitrage opportunity.
\end{answer}

\end{document}
