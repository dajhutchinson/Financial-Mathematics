\documentclass[11pt,a4paper]{article}

\usepackage[margin=1in, paperwidth=8.3in, paperheight=11.7in]{geometry}
\usepackage{amsmath,amsfonts,fancyhdr,bbm}
\usepackage[section,nohyphen]{DomH}
\headertitle{Finance Mathematics - Problem Sheet 4}

\begin{document}

\questionsfalse
\answersfalse

\title{Finance Mathematics - Problem Sheet 4}
\author{Dom Hutchinson}
\date{\today}
\maketitle


\begin{question}{1)}
  An urn contains red and black balls. At time-point 0 there are $a>0$ red and $b>0$ black blacks. At each time-point we randomly take a ball out of the urn and put it back, along with $c>0$ more balls of the same colour. Let $\{X_n\}_{n\in\nats_0}$ be the proportion of red balls at time point $n$.
  \par Show that $\{X_n\}_{n\in\nats_0}$ is a \textit{Martingale} wrt its \textit{Natural Filtration}
\end{question}

\begin{answer}{1)}
  Let $\omega_t$ denote the event that a red ball was taken from the bag in time-period $t$, meaning $\prob[\omega_t]=X_t$, and $R_t$ denote the number of red balls in the urn at the start of time-period $t$.
  \par Note that we can restate $X_t$ as
  \[ X_t=\frac{R_t}{a+b+tc} \]
  Now consider the conditional expectation of $X_t$ wrt its natural filtration
  \[\begin{array}{rcl}
    \expect[X_t|\mathcal{F}_{t-1}]&=&\expect\left[\frac{R_t}{a+b+tc}\bigg|\mathcal{F}_{t-1}\right]\\
    &=&\frac1{a+b+tc}\expect[R_t|\mathcal{F}_{t-1}]\\
    &=&\frac1{a+b+tc}\expect[R_{t-1}+c\indexed\{\omega_{t-1}\}|\mathcal{F}_{t-1}]\\
    &=&\frac1{a+b+tc}(R_{t-1}+cX_{t-1})\\
    &=&\frac1{a+b+tc}\left(R_{t-1}+c\frac{cR_{t-1}}{a+b+(t-1)c}\right)\\
    &=&\frac1{a+b+tc}\cdot\frac{R_{t-1}(a+b+(t-1)c+c)}{a+b+(t-1)c}\\
    &=&\frac1{a+b+tc}\cdot\frac{R_{t-1}(a+b+tc)}{a+b+(t-1)c}\\
    &=&\frac{R_{t-1}}{a+b+(t-1)c}\\
    &=&X_{t-1}\\
    \implies\expect[X_t|\mathcal{F}_{t-1}]&=&X_{t-1}
  \end{array}\]
  This is the definition of $\{X_t\}_{t\in\nats_0}$ being a \textit{Martingale}.
\end{answer}

\begin{question}{2) a)}
  Let $\{X_i\}_{i\in\nats}$ be IID random variables with moment generating function $\psi$. Show that the process $\{S_n\}_{n\in\nats_0}$ is a \textit{Martingale}
  \[ S_n:=\exp\left\{\lambda\sum_{j=1}^nX_j-n\ln(\psi(\lambda))\right\} \]
  Note that $\psi(\lambda):=\expect[e^{\lambda X}]$.
\end{question}

\begin{answer}{2) a)}
  \everymath={\displaystyle}
  \[\begin{array}{rcl}
    \expect[S_n|\mathcal{F}_{n-1}]&=&\expect[\exp\left\{\lambda\sum_{j=1}^nX_j-n\ln(\psi(\lambda))\right\}|\mathcal{F}_{n-1}]\\
    &=&\expect\left[\frac{\exp\left\{\lambda\sum_{j=1}^nX_j\right\}}{\psi(\lambda)^n}\bigg|\mathcal{F}_{n-1}\right]\\
    &=&\expect\left[\frac{\exp\left\{\lambda\sum_{j=1}^{n-1}X_j\right\}\cdot\exp\{\lambda X\}}{\psi(\lambda)^{n-1}\psi(\lambda)}\bigg|\mathcal{F}_{n-1}\right]\\
    &=&\frac{\exp\left\{\lambda\sum_{j=1}^{n-1}X_j\right\}}{\psi(\lambda)^{n-1}}\expect\left[\frac{\exp\{\lambda X\}}{\psi(\lambda)}\bigg|\mathcal{F}_{n-1}\right]\\
    &=&S_{n-1}\expect\left[\frac{\exp\{\lambda X\}}{\expect[\exp\{\lambda X\}]}\bigg|\mathcal{F}_{n-1}\right]\\
    &=&S_{n-1}\cdot1\\
    &=&S_{n-1}\\
    \implies\expect[S_n|\mathcal{F}_{n-1}]&=&S_{n-1}
  \end{array}\]
  This is the definition of $\{S_n\}_{n\in\nats_0}$ being a \textit{Martingale}.
\end{answer}

\begin{question}{2) b)}
  Let $\{Y_i\}_{i\in\nats}$ be IID random variables st that $\expect[Y_i]=0,\ \var(Y_i)=\sigma^2$ and $\mathcal{F}_i$ is the \textit{$\sigma$-algebra} generated by $\{Y_i\}_{i\in\nats}$.
  \par Show that $\{X_i\}_{i\in\nats}$ is a \textit{Martingale} wrt \textit{Filtration} $\mathcal{F}_i$.
  \[ X_i:=\left(\sum_{j=1}^iY_j\right)^2-n\sigma^2 \]
\end{question}

\begin{answer}{2) b)}
  \everymath={\displaystyle}
  \[\begin{array}{rcl}
    \expect[X_n|\mathcal{F}_{n-1}]&=&\expect\left[\left(\sum_{i=1}^nY_i\right)^2-n\sigma^2|\mathcal{F}_{n-1}\right]\\
    &=&\expect\left[\left(\sum_{i=1}^{n-1}Y_i+Y_n\right)^2|\mathcal{F}_{n-1}\right]-n\sigma^2\\
    &=&\expect\left[\left(\sum_{i=1}^{n-1}Y_i\right)^2+2Y_n\left(\sum_{i=1}^{n-1}Y_i+Y_n^2\right)|\mathcal{F}_{n-1}\right]-n\sigma^2\\
    &=&\left(\sum_{i=1}^{n-1}Y_i\right)^2+2\expect[Y_n|\mathcal{F}_{n-1}]\sum_{i=1}^{n-1}Y_i+\expect[Y_n^2|\mathcal{F}_{n-1}]-n\sigma^2\\
    &=&\left(\sum_{i=1}^{n-1}Y_i\right)^2+2\expect[Y_n]\sum_{i=1}^{n-1}Y_i+\expect[Y_n^2]-n\sigma^2\\
    &=&\left(\sum_{i=1}^{n-1}Y_i\right)^2+2\cdot0\cdot\sum_{i=1}^{n-1}Y_i+\left(\expect[Y_n^2]-0^2\right)-n\sigma^2\\
    &=&\left(\sum_{i=1}^{n-1}Y_i\right)^2+\left(\expect[Y_n^2]-\expect[Y_n]^2\right)-n\sigma^2\\
    &=&\left(\sum_{i=1}^{n-1}Y_i\right)^2+\sigma^2-n\sigma^2\\
    &=&\left(\sum_{i=1}^{n-1}Y_i\right)^2-(n-1)\sigma^2\\
    &=&X_{n-1}
  \end{array}\]
  This is the definition of $\{X_n\}_{n\in\nats_0}$ being a \textit{Martingale} wrt the natural filtration of $\{Y_n\}_{n\in\nats}$
\end{answer}

\begin{question}{3)}
  Let $\{X_t\}_{t\geq0}$ be a simple random walk with parameter $p\neq1/2$ and $\mathcal{F}_t$ be the $\sigma$-algebra generated by $\{X_t\}_{t\geq0}$.
  \par Define the processes $\{L_t\}_{t\geq0}$ and $\{M_t\}_{t\geq0}$ as
  \[\begin{array}{rcl}
    L_0&=&1\\
    L_t&=&\left(\frac{1-p}p\right)^{X_t}\\
    M_t&=&X_t-t(2p-1)
  \end{array}\]
  Show that $\{L_t\}_{t\geq0}$ and $\{M_t\}_{t\geq0}$ are both \textit{Martingales} wrt the natural filtration of $\{X_t\}_{t\geq0}$.
\end{question}

\begin{answer}{3)}
  \everymath={\displaystyle}
  Consider the conditional expectation of $L_t$ wrt the natural filtration $\mathcal{F}_t$ of $\{X_t\}_{t\geq0}$.
  \[\begin{array}{rcl}
    \expect[L_t|\mathcal{F}_{t-1}]&=&\expect\left[\left(\frac{1-p}p\right)^{X_t}\bigg|\mathcal{F}_{t-1}\right]\\
    &=&\expect\left[\left(\frac{1-p}p\right)^{X_{t-1}+Y_t}\bigg|\mathcal{F}_{t-1}\right]\\
    &=&\expect\left[\left(\frac{1-p}p\right)^{X_{t-1}}\cdot\left(\frac{1-p}p\right)^{Y_t}\bigg|\mathcal{F}_{t-1}\right]\\
    &=&\left(\frac{1-p}p\right)^{X_{t-1}}\expect\left[\left(\frac{1-p}p\right)^{Y_t}\bigg|\mathcal{F}_{t-1}\right]\\
    &=&\left(\frac{1-p}p\right)^{X_{t-1}}\expect\left[\left(\frac{1-p}p\right)^{Y_t}\right]\\
    &=&\left(\frac{1-p}p\right)^{X_{t-1}}\cdot\left\{\left(\frac{1-p}p\right)^1p+\left(\frac{1-p}p\right)^{-1}(1-p)\right\}\\
    &=&\left(\frac{1-p}p\right)^{X_{t-1}}\cdot\left\{1-p+p\right\}\\
    &=&\left(\frac{1-p}p\right)^{X_{t-1}}\\
    &=&L_{t-1}
  \end{array}\]
  This is the definition of $\{L_t\}_{t\geq0}$ being a \textit{Martingale}.
  \par Consider the conditional expectation of $M_t$ wrt the natural filtration $\mathcal{F}_t$ of $\{X_t\}_{t\geq0}$.
  \[\begin{array}{rcl}
    \expect[M_t|\mathcal{F}_{t-1}]&=&\expect[X_t-t(2p-1)|\mathcal{F}_{t-1}]\\
    &=&\expect[X_{t-1}+Y_t-t(2p-1)|\mathcal{F}_{t-1}]\\
    &=&X_{t-1}-t(2p-1)+\expect[Y_t|\mathcal{F}_{t-1}]\\
    &=&X_{t-1}-t(2p-1)+\expect[Y_t]\\
    &=&X_{t-1}-t(2p-1)+(2p-1)\\
    &=&X_{t-1}-(t-1)(2p-1)\\
    &=&M_{t-1}
  \end{array}\]
  This is the definition of $\{M_t\}_{t\geq0}$ being a \textit{Martingale}.
\end{answer}

\begin{question}{4)}
  Consider two  positive supermartingales $\{X_n^{(1)}\}_{n\in\nats},\{X_n^{(2)}\}_{n\in\nats}$ and a stopping time $\nu$ st $X_\nu^{(1)}\geq X_\nu^{(2)}$ for all $\omega\in\Omega$ where $\{\nu<\infty\}$.
  \par Show that the process $\{Y_n\}_{n\in\nats}$ is a positive supermartingale.
  \[ Y_n(\omega)=\begin{cases}
    X_n^{(1)}(\omega)&\text{if }n\leq\nu(\omega)\\
    X_n^{(2)}(\omega)&\text{if }n\geq\nu(\omega)
  \end{cases} \]
\end{question}

\begin{answer}{4)}
  Consider the following cases for the conditional expectation of $Y_n$ wrt its natural filtration $\mathcal{F}_n$.
  \begin{enumerate}
    \item Case $n<\nu(\omega)$.
    \[\begin{array}{rcll}
      \expect[Y_n(\omega)|\mathcal{F}_{n-1}]&=&\expect\left[X_n^{(1)}(\omega)|\mathcal{F}_{n-1}\right]&\text{by def }Y_n(\omega)\\
      &\leq&X_{n-1}^{(1)}(\omega)&\text{since }X_n^{(1)}\text{ is a supermartingale}\\
      &=&Y_{n-1}(\omega)
    \end{array}\]
    \item Case $n>\nu(\omega)$.
    \[\begin{array}{rcll}
      \expect[Y_n(\omega)|\mathcal{F}_{n-1}]&=&\expect\left[X_n^{(2)}(\omega)|\mathcal{F}_{n-1}\right]&\text{by def }Y_n(\omega)\\
      &\leq&X_{n-1}^{(2)}(\omega)&\text{since }X_n^{(2)}\text{ is a supermartingale}\\
      &=&Y_{n-1}(\omega)
    \end{array}\]
    It is worth noting that as $n>\nu(\omega)$ then $(n-1)>\nu(\omega)$, so the last equality is sound.
    \item Case $n=\nu(\omega)$.
    \[\begin{array}{rcll}
      \expect[Y_\nu(\omega)|\mathcal{F}_{\nu-1}]&=&\expect[X_\nu^{(2)}|\mathcal{F}_{\nu-1}]\\
      &\leq&X_{\nu-1}^{(2)}\\
      &\leq&X_{\nu-1}^{(1)}&\text{by def }\nu\\
      &=&Y_{\nu-1}(\omega)
    \end{array}\]
  \end{enumerate}
  In all cases $Y_n$ fulfils the condition for it to be a supermartingale.
\end{answer}

\end{document}
