\documentclass[11pt,a4paper]{article}

\usepackage[margin=1in, paperwidth=8.3in, paperheight=11.7in]{geometry}
\usepackage{amsmath,amsfonts,fancyhdr,bbm}
\usepackage[section,nohyphen]{DomH}
\headertitle{Financial Mathematics - Assessed Problem Sheet 1}

\begin{document}

\questionsfalse
% \answersfalse

\title{Financial Mathematics - Assessed Problem Sheet 1}
\author{Dom Hutchinson}
\date{\today}
\maketitle

\begin{question}{1. a)}
  The current price for AKOCOM shares is £37.50 per share. A put option on 1 share of AKOCOM maturing in exactly half a year, with a strike price of £30, is traded at £11.
  \par The price of the corresponding call option, with the same strike price and maturity date, is currently £18.50.
  \par Assume that the interest rate for lending money as well as positing money on a bank account is $5\%$ per annum.
\end{question}

\begin{question}{1. a) i.}
  Use the put-call parity to show that there exists arbitrage opportunities in the model above.
\end{question}

\begin{answer}{1. a) i.}
  From the question we know the strike price $K=\pounds30$, the annual interest rate is $r=0.05$, initial value of AKOCOM $S_0=\pounds37.5$, the initial price of the put option $P_0=\pounds11$ and the initial price of the call option $C_0=\pounds18.50$. The strike date is $T=1/2$, as the length of each option is 6 months, but we are given the annual interest rate.
  \par The put-call parity states that if $S_t+P_t-C_t=Ke^{r(T-t)}$ for all $t\in[0,T]$ then no arbitrage opportunities exist.
  \par Consider each side of this expression at time $t=0$
  \[\begin{array}{rcl}
    S_0+P_0-C_0&=&\pounds37.50+\pounds11-\pounds18.50\\
    &=&\pounds30\\
    Ke^{-r(T-0)}&=&30e^{-(0.05)(0.5-0)}\\
    &=&30e^{-0.025}\\
    &=&\pounds29.26\\
    \implies S_0+P_0-C_0&\neq&Ke^{-r(T-0)}
  \end{array}\]
  This shows that the put-call parity does not hold at time $t=0$ and thus an arbitrage opportunity exists.
\end{answer}

\begin{question}{1. a) ii.}
  Describe explicitly what actions an arbitrageur would carry out today and in six months time to attain riskless profit. You can assume that the arbitrageur does not have to pay any fees and that short selling is possible.
\end{question}

\begin{answer}{1. a) ii.}
  Today, $t=0$, the arbitrageur should do the following
  \begin{itemize}
    \item Short sell a share of AKOCOM, receiving £37.50.
    \item Take a long position on a call option, costing £18.50.
    \item Take a short position on a put option, receiving £11.
    \item Invest the net amount receiving from these transactions, $B_0=37.5-18.5+11=\pounds30$.
  \end{itemize}
  In six months time, $t=T=1/2$, our arbitrageur's bank balance will be $B_1=B_0e^{T/20}$.
  \par Let $S_T$ be the price of AKOCOM shares at time $T$. The arbitrageur should do the following
  \begin{itemize}
    \item If $S_T\geq\pounds30$ then exercise their call option. This costs £30 and fulfils the arbitrageur's short position on AKOCOM. The gains in this scenario are
    \[ B_1-30=30e^{T/20}-30>0 \]
    The holder of long position in our put option will not exercise their option in this scenario as they would loose money.
    \item If $S_T<\pounds30$ then tear-up the call option. The holder of the long position in our put option will exercise their option in this scenario as they will make a profit. This means we have to buy a share of AKOCOM from them for £30, this fulfils our short position on AKOCOM. Our profit in this scenario is again
    \[ B_1-30=30e^{T/20}-30>0 \]
  \end{itemize}
  These two scenarios cover all outcomes in six months time, and show that our arbitrageur makes a risk-free profit under both scenarios.
\end{answer}

\begin{question}{1. b)}
  Let $S_t$ be the time $t$ price of some stock and $C_t^1$ and $C_t^2$ the time $t$ prices of European call options on that stock, with maturity $T$ and strike prices $K_1$ and $K_2$, with $K_1<K_2$. We also consider a \textit{capped} call option $X$ on the same stock with same maturity $T$ and value
  \[ X_T=\begin{cases}K_2-K_1&\text{if }K_2\leq S_T\\S_T-K_1&\text{if }S_T\in[K_1,K_2]\\0&\text{if }S_T\leq K_1\end{cases} \]
  Find the time $t$ price $X_t$ of the capped call option in terms of $C_t^1$ and $C_t^2$.
\end{question}

\begin{answer}{1. b)}
  Consider the payouts from the two European call options at time $t=T$
  \[\begin{array}{rcl}
    C_T^1&=&\{S_T-K_1\}_+\\
    C_T^2&=&\{S_T-K_2\}_+
  \end{array}\]
  where $\{x\}_+:=\max\{0,x\}$. Note that since $K_1<K_2$ then $C_T^2\leq C_T^1\ \forall\ S_T\in\reals^{\geq0}$.
  \par We can restate $X_T$ in terms of $C_T^1$ and $C_T^2$ as
  \[\begin{array}{rcl}
    X_T&=&\begin{cases}
      C_T^1-C_T^2&\text{if }S_T\geq K_2\\
      C_T^1&\text{if }S_T\in[K_1,K_2]\\
      0&\text{if }S_T\leq K_1\\
    \end{cases}\\
    &=&C_T^1-C_T^2
  \end{array}\]
  Using the ``No-Arbitrage Principle'' it can be shown that if two, or more, financial derivatives have the same value at time $T$, then their prices will coincide at all times $t<T$. This means the fair price at time $t$ for this capped call option is
  \[ X_t=C_t^1-C_t^2 \]
\end{answer}

\begin{question}{1. c)}
  Consider a single-period model with one stock that has a current price of $S_0=1$ and a future price of either $S_1(\omega_1)=1.3$ or $S_1(\omega_2)=1.1$. The interest rate of the risk-free asset is $r$.
\end{question}

\begin{question}{1. c) i.}
  Find all values for $r$ such that there exists a risk neutral probability measure, and determine that measure in terms of $r$.
\end{question}

\begin{answer}{1. c) i.}
  A probability measure $\Q$ is a \textit{Risk-Neutral Probability Measure} if the following all hold
  \begin{enumerate}
    \item $\Q(\{\omega\})>0\ \forall\ \omega\in\Omega$; and,
    \item $\expect_\Q[S_1^*(1)]=S_1^*(0)$
  \end{enumerate}
  Additionally, as $\Q$ is a probability measure we have $\sum_{\omega\in\Omega}\Q(\{\omega\})=1$.
  \par From the question we have that $S_0=1,\ S_1(\omega_1)=1.3$ and $S_2(\omega_2)=1.1$. Let $r$ denote the risk-free interest rate, $q_1:=\Q(\{\omega_1\})$ and $q_2:=\Q(\{\omega_2\})$. Note that the Bank process at time $t=1$ has value $B_1=1+r$.
  \par Under the conditions of this question, we can derive the follow equations which must hold in order for $\Q$ to be a \textit{Risk-Neutral Probability Measure}
  \[\begin{array}{rrclcl}
    &q_1+q_2&=&1&\quad&(1)\\
    \text{and }&\expect_\Q[S_1^*(1)]&=&S_1^*(0)\\
    \implies&q_1S_1^*(\omega_1)+q_2S_1^*(\omega_2)&=&\frac{S_1(0)}{B_0}\\
    \implies&q_1\cdot\frac{S_1(\omega_1)}{B_1}+q_2\cdot\frac{S_1(\omega_2)}{B_1}&=&\frac{S_1(0)}{B_0}\\
    \implies&q_1\cdot\frac{1.1}{1+r}+q_2\cdot\frac{1.3}{1+r}&=&1&&(2)\\
  \end{array}\]
  \par From equations $(1),(2)$ we can deduce values for $q_1,q_2$ in terms of $r$.
  \[\begin{array}{rrcl}
    &q_2&=&1-q_1\\
    \implies&q_1\cdot\frac{1.1}{1+r}+(1-q_1)\cdot\frac{1.3}{1+r}&=&1\\
    \implies&\frac{13}{10(1+r)}-q_1\cdot\frac2{10(1+r)}&=&1\\
    \implies&q_1&=&\frac{13-10(1+r)}2\\
    &&=&\frac{3-10r}2\\
    \implies&q_2&=&1-\frac{3-10r}2\\
    &&=&\frac{10r-1}{2}
  \end{array}\]
  Thus, probability measure $\Q$ can be stated as
  \[ \Q(\{\omega_1\})=\frac{3-10r}{2}\quad\Q(\{\omega_2\})=\frac{10r-1}2 \]
  As both these quantities must take values in $[0,1]$ we can deduce the range of interest rates $r$ where a \textit{Risk-Neutral Probability Measure} exists.
  \[\begin{array}{rrclcrcl}
    &\Q(\{\omega_1\})&\in&[0,1]&\quad&\Q(\{\omega_2\})&\in&[0,1]\\
    \implies&\frac{3-10r}{2}&\in&[0,1]&\quad&\frac{10r-1}2&\in&[0,1]\\
    \implies&3-10r&\in&[0,2]&\quad&10r-1&\in&[0,2]\\
    \implies&10r&\in&[1,3]&\quad&10r&\in&[1,3]\\
    \implies&r&\in&[0.1,0.3]&\quad&r&\in&[0.1,0.3]
  \end{array}\]
  Thus, there exists a \textit{Risk-Neutral Probability Measure} if $r\in[0.1,0.3]$.
\end{answer}

\begin{question}{1. c) ii.}
  Assume that $r=0.4$ and that stock holders are getting paid a dividend of 0.1 at time point 1. Describe explicitly what an arbitrageur would have to do at time point 0 and at time point 1 to gain a riskless profit, and calculate the profit in each state $\omega_1,\omega_2$. Again, assume that the arbitrageur does not have to pay any fees and that short-selling is possible.
\end{question}

\begin{answer}{1. c) ii.}
  At time-point $t=0$ the agent should do the following
  \begin{itemize}
    \item Short a share, receiving $S_0=$£1.
    \item Invest this £1 in the risk-free asset.
  \end{itemize}
  At time-point $t=1$ this investment is work $B_1=1+r=1.4$.
  Then, at time-point $t=1$ the agent should do the following.
  \begin{itemize}
    \item Pay the dividend $D=0.1$ to the agent who leant us the share used for our short position.
    \item Buy a share, at whatever the current price is, in order to fulfil our short position.
  \end{itemize}
  If our arbitrageur follows this strategy and event $\omega_1$ occurs, then they make a risk-free profit of
  \[ B_1-S_1(\omega_1)-D=1.4-1.3-0.1=0 \]
  \par If our arbitrageur follows this strategy and event $\omega_2$ occurs, then they make a risk-free profit of
  \[ B_1-S_1(\omega_2)-D=1.4-1.1-0.1=0.2 \]
  Thus, under either event our arbitrageur is guaranteed not to loose money.
\end{answer}

\end{document}
