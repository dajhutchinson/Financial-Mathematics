\documentclass[11pt,a4paper]{article}

\usepackage[margin=1in, paperwidth=8.3in, paperheight=11.7in]{geometry}
\usepackage{amsmath,amsfonts,fancyhdr,bbm}
\usepackage[section,nohyphen]{DomH}
\headertitle{Financial Mathematics - Problem Sheet 1}

\begin{document}

\questionsfalse
% \answersfalse

\title{Financial Mathematics - Problem Sheet 1}
\author{Dom Hutchinson}
\date{\today}
\maketitle


\begin{question}{1)}
  Consider a share which trades on Jan 1st at £8. Suppose we know that on July 1st it will be worth £16 with probability $q=2/3$ or £4 with probability $1/3$.
  \par On Jan 1st, the following contract is freely traded on the market:
  \begin{quote}
    ``If the share price goes up, I will pay you £2. If the share goes down, I will pay you £1.''
  \end{quote}
  You have access to a bank account with pays 10\% interest over the six-month period.
  \par Consider a portfolio on Jan 1st which contains $H_0$ pounds in the bank, and $H_1$ units of stock. Show that we can find values for $H_0$ and $H_1$ such that the portfolio is worth the same as the contract on July 1st, regardless of whether the share goes up or down.
  \par How much is the portfolio worth on Jan 1st? And, how much should the contract be worth on Jan 1st?
\end{question}

\begin{answer}{1)}
  From the question we can define the following two equations, equating the values of the contract \& portfolio on July 1st for each of the outcomes
  \[\begin{array}{rrcl}
    \text{(Share Increases)}&\frac{11}{10}H_0+16H_1&=&2\\
    \text{(Share Decreases)}&\frac{11}{10}H_0+4H_1&=&1
  \end{array}\]
  This system of equations can be solved as follows
  \[\begin{array}{rrcl}
    &\left(\frac{11}{10}-\frac{11}{10}\right)H_0+(16-4)H_1&=&(2-1)\\
    \implies&12H_1&=&1\\
    \implies&H_1&=&\frac1{12}\\\\
    \implies&\frac{11}{10}H_0+16\cdot\frac1{12}&=&2\\
    \implies&H_0&=&\frac{20}{33}
  \end{array}\]
  On Jan 1st this portfolio is worth
  \[ H_0+8H_1=\frac{20}{33}+8\cdot\frac1{12}=\frac{14}{11} \]
  Thus, the contract is worth £$14/11\simeq$£1.27 on Jan 1st.
\end{answer}

\begin{question}{2)}
  Continue in the setting of \texttt{Q 1)}. Let $S^*$ be the share price normalised by the value accumulated by placing £1 in the bank on 1st Jan. What is the expected value of $S^*$ on July 1st as a function of $q$? (The probability of the share price increasing).
  \par For what value $q^*$ is this $\expect[S^*]=8$, the initial share value?
  \par What is the expected value of the portfolio found in \texttt{Q 1)} on July 1st, using the value $q^*$?
  \par How does this relate to the Jan 1st value found in the previous question?
\end{question}

\begin{answer}{2)}
  We have that $S^*=S/(1+r)$. By the question $r=.1$, so $S*=\frac{11}{10}S$.\\
  Thus, the expected value of $S^*$ on July 1st, given variable $q$, is
  \[\begin{array}{rcl}
    \expect[S^*;q]&=&\frac{10}{11}(16q+4(1-q))\\
    &=&\frac{10}{11}(12q+4)
  \end{array}\]
  We find the probability $q^*$ which produces an expected value of 8 as follows.
  \[\begin{array}{rrcl}
    &\frac{10}{11}(12q^*+4)&=&8\\
    \implies&12q^*+4&=&\frac{88}{10}\\
    \implies&12q^*&=&\frac{48}{10}\\
    \implies&q^*&=&\frac{4}{11}
  \end{array}\]
  With this probability, the portfolio derived in \texttt{Q 1)} has the expected following value on July 1st
  \[\begin{array}{rcl}
    \expect[\text{Portfolio Value}]&=&\frac{11}{10}H_0+\left\{(q^*\cdot H_1\cdot 16+(1-q^*)\cdot H_1\cdot4\right\}\\
    &=&\frac{11}{10}\cdot\frac{20}{33}+\left\{\frac4{11}\cdot\frac1{12}\cdot16+\frac{7}{11}\cdot\frac1{12}\cdot4\right\}\\
    &=&\frac{15}{11}
  \end{array}\]
  This is greater than the value of the portfolio on Jan 1st, as calculated in \texttt{Q 1)}. So, even if the normalised price of the share is not expected to increase between Jan 1st and July 1st, the value of the portfolio is expected to increase.
\end{answer}

\begin{question}{3a)}
  Eight horses take part in a horse race. The odds offered by a bookmaker are 1:1 for horse 1, 2:1 for horse 2 and 50:1 for horses 3-8. Are there any arbitrage communities?
  \par Note that a £$M$ bet with odds of $N:M$ pays out £$N+M$ if the horse wins, and £0 if it loses.
\end{question}

\begin{answer}{3a)}
  Consider the minimum proportion of our total stake needed to be placed on each horse in order to cover the costs of our losses if that horse wins.
  \begin{center}
    \begin{tabular}{c|c|c|c}
      &Horse \#1&Horse \#2&Horses \#3-\#6\\\hline
      Required Pct.&.5\%&.3333&.0196\%
    \end{tabular}
  \end{center}
  This sums to a total of 0.9509\%. This means that we do not require our total stake in order to cover any loses, regardless of outcome (ie An arbitrage opportunity exists).
  \par Distributing our total stake between each horse in the ratio of the proportions given in the above table will produce an arbitrage. Here are the details
  \begin{center}
    \begin{tabular}{c|c|c|c}
      &Horse \#1&Horse \#2&Horses \#3-\#6\\\hline
      \textit{Stake}&£50&£33.33&£1.96\\\hline
      \textit{Returns}&£100&£99.99&£99.96\\
      \textit{Losses}&£45.09&£61.76&£93.13\\\hline
      \textit{Profit}&£54.91&£38.23&£6.83\\\hline
    \end{tabular}
  \end{center}
  \par \textit{Returns} \& \textit{Losses} are the flows of money if that horse wins the race. As you can see from the table, the returns are greater than the losses in all outcomes. Naturally, a risk-free profit is made for any scaling of this set of stakes.
\end{answer}

\begin{question}{3b)}
  Generalise your answer to \texttt{Q 3a)} to the case of $n$ horses and derive a condition under which there are no arbitrage opportunities.
\end{question}

\begin{answer}{3b)}
  Consider the scenario in \texttt{Q 3a)}, but with the number of horses whose odds are offered at 50:1 now being a variable $n\in\nats$.
  \par Arbitrage opportunities do not exist when the sum of minimum proportions of our total stake required for the winnings from each horse to cover the losses from the other horses exceeds 100\%. We can find such an $n$ as follows
  \[\begin{array}{rrcl}
    &\frac12+\frac13+\frac{n}{51}&>&1\\
    \implies&\frac{n}{51}&>&\frac16\\
    \implies&n&>&\frac{51}6\\
    \implies&n&\geq&9\text{ as }n\in\nats
  \end{array}\]
  So, if one horse has odds 1:1, another has odds 2:1, and 9, or more, have odds 50:1 then no arbitrage opportunity exists.
\end{answer}

\end{document}
