\documentclass[11pt,a4paper]{article}

\usepackage[margin=1in, paperwidth=8.3in, paperheight=11.7in]{geometry}
\usepackage{amsmath,amsfonts,fancyhdr,bbm}
\usepackage[section,nohyphen]{DomH}
\headertitle{Financial Mathematics - Problem Sheet 3}

\begin{document}

% \questionsfalse
% \answersfalse

\title{Financial Mathematics - Problem Sheet 3}
\author{Dom Hutchinson}
\date{\today}
\maketitle


\begin{question}{1.}
  The following table shows the price process of a certain stock at three successive time points
  \begin{center}
    \begin{tabular}{c|cccccccc}
      $t$&$S(t,\omega_1)$&$S(t,\omega_2)$&$S(t,\omega_3)$&$S(t,\omega_4)$&$S(t,\omega_5)$&$S(t,\omega_6)$&$S(t,\omega_7)$&$S(t,\omega_8)$\\\hline
      0&10&10&10&10&10&10&10&10\\
      1&15&15&15&10&10&9&9&9\\
      2&19&18&14&11&9&12&13&8
    \end{tabular}
  \end{center}
\end{question}

\begin{question}{1. a)}
  Identify for all three time points the corresponding partitions $\mathcal{P}_0,\mathcal{P}_1,\mathcal{P}_2$ of $\Omega=\{\omega_1,\dots,\omega_8\}$.
\end{question}

\begin{answer}{1. a)}
  \[\begin{array}{rcl}
    \mathcal{P}_0&=&\big\{\Omega\big\}\\
    \mathcal{P}_1&=&\big\{\{\omega_1,\omega_2,\omega_3\},\{\omega_4,\omega_5\},\{\omega_6,\omega_7,\omega_8\}\big\}\\
    \mathcal{P}_2&=&\big\{\{\omega_1\},\{\omega_2\},\{\omega_3\},\{\omega_4\},\{\omega_5\},\{\omega_6\},\{\omega_7\},\{\omega_8\}\big\}
  \end{array}\]
\end{answer}

\begin{question}{1. b)}
  Write down explicitly the corresponding $\sigma$-algebras $\mathcal{F}_0,\mathcal{F}_1,\mathcal{F}_2$.
\end{question}

\begin{answer}{1. b)}
  \[\begin{array}{rcl}
    \mathcal{F}_0&=&\big\{\emptyset,\Omega\big\}\\
    \mathcal{F}_1&=&\big\{\emptyset,\Omega,\{\omega_1,\omega_2,\omega_3\},\{\omega_4,\omega_5,\omega_6,\omega_7,\omega_8\},\\
    &&\{\omega_4,\omega_5\},\{\omega_1,\omega_2,\omega_3,\omega_6,\omega_7,\omega_8\},\{\omega_6,\omega_7,\omega_8\},\\
    &&\{\omega_1,\omega_2,\omega_3,\omega_4,\omega_5\}\big\}\\
    \mathcal{F}_2&=&2^\Omega
  \end{array}\]
\end{answer}

\begin{question}{2.}
  Consider a single-period model with two risky securities $S_1$ and $S_2$. We assume that there is a bank account with a risk-free interest rate $r=1/9$ and the price process is given by the following table
  \begin{center}
    \begin{tabular}{c|c|cccc}
      $n$&$S_n(0)$&$S_n(1)(\omega_1)$&$S_n(1)(\omega_2)$&$S_n(1)(\omega_3)$&$S_n(1)(\omega_4)$\\\hline
      1&15&20&20&40/3&20/3\\
      2&30&40&80/3&80/3&20/3
    \end{tabular}
  \end{center}
\end{question}

\begin{question}{2. a)}
  Specify the value and gains processes $V$ and $G$ for this model as well as their discounted versions $V^*$ and $G^*$.
\end{question}

\begin{answer}{2. a)}
  The value process $V_t(\omega)$ for this model is as follows
  \begin{center}
    \begin{tabular}{c|llll}
      $\omega\setminus t$&0&1\\\hline
      $\omega_1$&$H_0+15H_1+30H_2$&$(10/9)H_0+20H_1+40H_2$\\
      $\omega_2$&$H_0+15H_1+30H_2$&$(10/9)H_0+20H_1+(80/3)H_2$\\
      $\omega_3$&$H_0+15H_1+30H_2$&$(10/9)H_0+(40/3)H_1+(80/3)H_2$\\
      $\omega_4$&$H_0+15H_1+30H_2$&$(10/9)H_0+(20/3)H_1+(20/3)H_2$
    \end{tabular}
  \end{center}
  and the discounted value process $V_t^*(\omega)=V_t(\omega)/B_t$ for this model is
  \begin{center}
    \begin{tabular}{c|llll}
      $\omega\setminus t$&0&1\\\hline
      $\omega_1$&$H_0+15H_1+30H_2$&$H_0+18H_1+36H_2$\\
      $\omega_2$&$H_0+15H_1+30H_2$&$H_0+18H_1+24H_2$\\
      $\omega_3$&$H_0+15H_1+30H_2$&$H_0+12H_1+24H_2$\\
      $\omega_4$&$H_0+15H_1+30H_2$&$H_0+6H_1+6H_2$
    \end{tabular}
  \end{center}
  The gains process $G(\omega)$ for this model is as follows
  \begin{center}
    \begin{tabular}{c|l}
      $\omega$&$G(\omega)$\\\hline
      $\omega_1$&$(1/9)H_0+5H_1+10H_2$\\
      $\omega_2$&$(1/9)H_0+5H_1-(10/3)H_2$\\
      $\omega_3$&$(1/9)H_0-(5/3)H_1-(10/3)H_2$\\
      $\omega_4$&$(1/9)H_0-(25/3)H_1-(70/3)H_2$
    \end{tabular}
  \end{center}
  and the discounted gains process $G^*(\omega)=\sum_{i=1}^2H_1\Delta S_i^*(\omega)$ for this model is as follows
  \begin{center}
    \begin{tabular}{c|l}
      $\omega$&$G^*(\omega)$\\\hline
      $\omega_1$&$3H_1+6H_2$\\
      $\omega_2$&$3H_1-6H_2$\\
      $\omega_3$&$-3H_1-6H_2$\\
      $\omega_4$&$-9H_1-24H_2$
    \end{tabular}
  \end{center}
\end{answer}

\begin{question}{2. b)}
  Specify $\mathbb{W}$ where
  \[ \mathbb{W}=\{X\in\reals^K:X=G^*\text{ for some trading strategy }H\} \]
  and using the fact that an arbitrage opportunity is a trading strategy which requires $G^*(\omega)\geq0$, determine the arbitrage opportunities in this model.
  \par Hint: Consider $G^*(\omega_1)$ and $G^*(\omega_3)$ first.
\end{question}

\begin{answer}{2. b)}
  Consider $X\in\reals^4$ st $X=G^*$. Then
  \[ \begin{pmatrix} X_1\\X_2\\X_3\\X_4\end{pmatrix}=\begin{pmatrix}3H_1+6H_2\\3H_1-6H_2\\-3H_1-6H_2\\-9H_1-24H_2\end{pmatrix} \]
  Thus
  \[ \mathbb{W}=\text{span}\left\{\begin{pmatrix}3\\3\\-3\\-9\end{pmatrix},\begin{pmatrix}6\\-6\\-6\\-24\end{pmatrix}\right\}=\text{span}\left\{\begin{pmatrix}1\\1\\-1\\-3\end{pmatrix},\begin{pmatrix}1\\-1\\-1\\-4\end{pmatrix}\right\} \]
  Consider the set $\mathbb{A}:=\{X\in\reals^4:x\geq0\ \forall\ x\in X\}$ and the elements of $\mathbb{W}\cap\mathbb{A}$. This elements form the set of possible gains which would produce a risk-free profit (i.e. an arbitrage opportunity).
  \[\begin{array}{rcl}
    \mathbb{W}\cap\mathbb{A}&=&\big\{X\in\reals^4:X_1=x_3=0;X_2,X_4>0\big\}\\
    X\in(\mathbb{W}\cap\mathbb{A})&\implies&X=\begin{pmatrix}0&\lambda&0&\mu\end{pmatrix}\text{ with }\lambda,\mu>0
  \end{array}\]
  How consider the $X$s where $X=G^*$ and $X\in(\mathbb{W}\cap\mathbb{A})$.
  \[\begin{array}{rrcl}
    &X_1=3H_1+6H_2&=&0\\
    \implies&H_1&=&-2H_2\\\\
    &X_2=3H_1-6H_2&=&\lambda\\
    \implies&-12H_2&=&\lambda\\
    \implies&H_2&=&-\frac1{12}\lambda\\
    \implies&H_1&=&\frac16\lambda\\\\
    &X_4=-9H_1-24H_2&=&\mu\\
    \implies&-6H_2&=&\mu\\
    \implies&H_2&=&-\frac16\mu\\
    \implies&H_1&=&\frac13\mu\\
    \implies&\mu&=&\frac12\lambda
  \end{array}\]
  Thus the trading strategies which exploit arbitrage opportunities in this model as those in the set
  \[ \left\{H\in\reals^3:H=\begin{pmatrix}0&\frac16\lambda&-\frac1{12}\lambda\end{pmatrix}:\lambda>0\right\} \]
\end{answer}

\begin{question}{2. c)}
  Briefly explain why there are no risk-neutral probability measures and why this model is not complete
\end{question}

\begin{answer}{2. c)}
  The ``No-Arbitrage Theorem'' states that risk-neutral probability measures only exists for a model \underline{iff} no arbitrage opportunities exist. In this model arbitrage opportunities do exist, thus no risk-neutral probability measures can.
  \par Consider matrix $A$ which summarises this model
  \[
  A=\begin{pmatrix}
    B_1(\omega_1)&S_1(1)(\omega_1)&S_2(1)(\omega_1)\\
    B_1(\omega_2)&S_1(1)(\omega_2)&S_2(1)(\omega_2)\\
    B_1(\omega_3)&S_1(1)(\omega_3)&S_2(1)(\omega_3)\\
    B_1(\omega_4)&S_1(1)(\omega_4)&S_2(1)(\omega_4)
  \end{pmatrix}==\begin{pmatrix}
    10/9&20&40\\
    10/9&20&80/3\\
    10/9&40/3&80/3\\
    10/9&20/3&20/3
  \end{pmatrix}
  \]
  Matrix $A$ only has three linearly-independent columns, thus $AH$ can only span $\reals^3$ for any trading strategy $H$. As $AH$ cannot span $\reals^4$, the market is \underline{not} complete.
\end{answer}

\begin{question}{3.}
  Consider a sample space $\Omega=\{\omega_1,\omega_2,\omega_3,\omega_4\}$.
\end{question}

\begin{question}{3. a)}
  Decide whether the following collections of subsets of $\Omega$ are $\sigma$-algebras or not:
  \[ \mathcal{F}_1=\big\{\emptyset,\{\omega_1,\omega_2\},\{\omega_3,\omega_4\},\Omega\big\}\quad\mathcal{F}_2=\big\{\emptyset,\{\omega_1\},\{\omega_2\},\{\omega_3\},\{\omega_4\}\big\} \]
\end{question}

\begin{answer}{3. a)}
  \par $\mathcal{F}_1$ is a $\sigma$-algebra.
  \par $\mathcal{F}_2$ is \underline{not} a $\sigma$-algebra as (among other reasons) $\{\omega_1\},\{\omega_2\}\in\mathcal{F}_2$ but $\{\omega_1,\omega_2\}\not\in\mathcal{F}_2$.
\end{answer}

\begin{question}{3. b)}
  Define a $\sigma$-algebra $\mathcal{F}$ and a function $\omega\to X(\omega)$ with values in the real numbers st $X$ is not measurable wrt $\mathcal{G}$.
\end{question}

\begin{answer}{3. b)}
  Consider the $\sigma$-algebra $\mathcal{G}:=\{\emptyset,\Omega\}$ and the function $X(\omega_i)=i$. $X$ is \underline{not} measurable wrt $\mathcal{G}$ as $\{\omega_1\},\{\omega_2\},\{\omega_3\},\{\omega_4\}\not\in\mathcal{G}$.
\end{answer}

\begin{question}{3. c)}
  Find a $\sigma$-algebra $\mathcal{H}$ st \underline{all} function $\omega\to X(\omega)$ are measurable wrt $\mathcal{H}$.
\end{question}

\begin{answer}{3. c)}
  The powerset of $\Omega$ contains all possible subsets of $\Omega$ and thus all functions $\omega\to X(\omega)$ are measurable wrt it to.
  \[ \mathcal{H}=2^\Omega \]
\end{answer}

\end{document}
